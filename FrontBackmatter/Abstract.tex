%*******************************************************
% Abstract
%*******************************************************
%\renewcommand{\abstractname}{Abstract}
\addcontentsline{toc}{chapter}{\abstractname}

\pdfbookmark[1]{Abstract}{Abstract}
\begingroup
\let\clearpage\relax
\let\cleardoublepage\relax
\let\cleardoublepage\relax

\chapter*{Abstract}

Indoor localization is a very important theme in the industry 4.0 era.
This thesis presents some of the main problems and solutions that regards this
topic, by also analyzing hybrid approaches combining different techniques.
Besides the generic overview of the topic and the presentation of the most common techniques
and the current literature review, in this thesis there are also results and methodologies of a
practical work with the goal of analyzing the utilization of Ultra-wideband (UWB) and WiFi
wireless protocols, together with localization techniques Received
signal strength (RSS) and two-way ranging (TWR), which is based on Time of flight (TOF).

The practical work is divided in two main parts: the first one consists in setting up a
system that is able to collect data by exploiting both technologies and using it in
some experiments that reproduce both Line-of-sight (LOS) and Non-line-of-sight (NLOS) conditions; the
second one is about the analysis of the data that had been collected, in
which Received signal strength indication (RSSI) WiFi is predicted through the data relative to UWB in order to study
the relation between the two protocols, RSSI values obtained with both UWB and
WiFi are compared in a visual way and localization is performed by using
information of both protocols.
The devices that were chosen in order to analyze the performances of UWB and
WiFi protocols are respectively the DW1000 Decawave chip and some simple
Raspberry Pi 3 model B.

The results obtained at the end of this work show that by combining the data that
were collected during the measurements process with the coordinates estimated by
the DecaWave system with Machine Learning techniques it is possible to improve
the precision of the Decawave system itself, up to double it in some scenarios.


\vfill
\newpage
\pdfbookmark[1]{Sommario}{Sommario}
\chapter*{Sommario}
La localizzazione in ambienti interni è una tematica molto importante nell'era
dell'industria 4.0.
Questa tesi presenta alcuni dei principali problemi e soluzioni che caratterizzano questo
argomento, analizzandone anche approcci ibridi che combinano più tecniche tra di loro.
La panoramica generale di questa tematica, la presentazione delle tecniche di uso
più comune e l'analisi dello stato attuale della letteratura sono
accompagnate da un lavoro pratico che ha l'obiettivo di analizzare l'utilizzo dei protocolli
wireless Ultra-wideband (UWB) e WiFi, affiancati alle tecniche di localizzazione
Received signal strength (RSS) e two-way ranging (TWR), la quale è basata su Time of flight (TOF).

Il lavoro pratico è diviso in due parti principali: la prima consiste nella messa in piedi
di un sistema che possa raccogliere dati sfruttando entrambe le tecnologie e
nell'impiego di quest'ultimo in esperimenti che riproducono sia condizioni di linea di vista che non;
la seconda consiste nell'analisi dei dati raccolti, in cui
viene predetto il Received signal strength indication (RSSI) del WiFi mediante i dati relativi ad UWB in modo da studiare
la relazione tra i due protocolli, vengono confrontati i valori di RSSI ottenuti
sia con UWB che con WiFi in maniera visiva e viene effettuata la
localizzazione usando le informazioni di entrambi i protocolli.
I dispositivi scelti per analizzare le prestazioni dei protocolli UWB e WiFi
sono rispettivamente il chip DW1000 della DecaWave e dei semplici Raspberry Pi 3 model B.

I risultati ottenuti alla fine di questo lavoro mostrano che combinando i dati
raccolti durante il processo di misurazioni con le coordinate stimate dal sistema
DecaWave mediante techiche di Machine Learning è possibile
migliorare la precisione del sistema DecaWave stesso, fino anche a raddoppiarla in alcuni scenari.

\endgroup
